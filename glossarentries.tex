%--------------------------------------------------------------------
%
%   Version: 3.25        (Koma-Skript, documentclass{scrbook})
%   Datum: 01.06.2018
%   Autor: Johannes Bier
%
%--------------------------------------------------------------------
% die Auflistung für das Abkürzungsverzeichnis 
%--------------------------------------------------------------------
\newacronym{DC}{DC}{Gleichstrom (Direct Current)}
\newglossaryentry{AC}{
	name={AC},
	description={Wechselstrom (Alternating Current)},
	first={\glsentrydesc{AC} (\glsentrytext{AC})},
	plural={ACs},
	descriptionplural={Wechselströme (Alternating Currents)},
	firstplural={\glsentrydescplural{AC} (\glsentryplural{AC})}
}
\newacronym{TA}{TA}{Taktperiode des Umrichtersystems}
\newacronym{ETI}{ETI}{Elektrotechnisches Institut}
\newglossaryentry{EPSR}{
	name={EPSR},
	description={Einplatinenstromrichter},
	first={\glsentrydesc{EPSR} (\glsentrytext{EPSR})},
	plural={EPSR},
	descriptionplural={Einplatinenstromrichter},
	firstplural={\glsentrydescplural{EPSR} (\glsentryplural{EPSR})}
}
\newacronym{EMV}{EMV}{Elektromagnetischen Verträglichkeit}
\newacronym{FPGA}{FPGA}{Field Programmable Gate Array}
\newglossaryentry{DSP}{
	name={DSP},
	description={digitaler Signalprozessor},
	first={\glsentrydesc{DSP} (\glsentrytext{DSP})},
	plural={DSP's},
	descriptionplural={digitale Signalprozessoren},
	firstplural={\glsentrydescplural{DSP} (\glsentryplural{DSP})}
} 
\newacronym{ETI-Bus}{ETI-Bus}{kommunikationsbus des DSP-Systems}
\newacronym{USB}{USB}{Universal Serial Bus}
\newacronym{MMC}{MMC}{Modularer-Multilevel-Umrichter}
\newacronym{M3C}{M3C}{Modularer-Multilevel-Matrix-Umrichter}
\newacronym{ESB}{ESB}{Ersatzschaltbild}
\newacronym{3AC}{3AC}{dreiphasiger Wechselstrom}
\newacronym{AFE}{AFE}{Active Front End}
\newglossaryentry{TSS}{
	name={TSS},
	description={Tiefsetzsteller},
	first={\glsentrydesc{TSS} (\glsentrytext{TSS})},
	plural={TSS},
	descriptionplural={Tiefsetzsteller},
	firstplural={\glsentrydescplural{TSS} (\glsentryplural{TSS})}
} 
\newacronym{LAM}{LAM}{Lastmaschine}
\newglossaryentry{DUT}{
	name={DUT},
	description={Prüfling (Device Under Test)},
	first={\glsentrydesc{DUT} (\glsentrytext{DUT})}
} 
\newacronym{AD}{AD}{Analog Digital}
\newacronym{ADC}{ADC}{Analog Digital Wandler}
\newacronym{HMK}{HMK}{Hochleistungsmodulatorkarte}
\newacronym{GM}{GM}{Gleichstrommaschine}
\newacronym{ASM}{ASM}{Asynchronmaschine}
\newacronym{PLL}{PLL}{Phase Look Loop}
\newacronym{FO}{FO}{Feldorientierten Regelung}
\newacronym{FIR}{FIR}{Finite Impuls Response - Filter mit endlicher Impulsantwort}
\newacronym{DG}{DG}{Diodengleichrichter}
\newacronym{PWM}{PWM}{Pulsweitenmodulation (Puls Wide Modulation)}
\newglossaryentry{HB}{
	name={HB},
	description={Halbbrücke},
	first={\glsentrydesc{HB} (\glsentrytext{HB})},
	plural={HBs},
	descriptionplural={Halbbrücken},
	firstplural={\glsentrydescplural{HB} (\glsentryplural{HB})}
} 
\newglossaryentry{LED}{
	name={LED},
	description={light emitting diode},
	first={\glsentrydesc{LED} (\glsentrytext{LED})},
	plural={LED's},
	descriptionplural={light emitting diodes},
	firstplural={\glsentrydescplural{LED} (\glsentryplural{LED})}
} 
\newglossaryentry{EEPROM}{
	name={EEPROM},
	description={electrically erasable programmable read-only memory},
	first={\glsentrydesc{EEPROM} (\glsentrytext{EEPROM})},
	plural={EEPROM's},
	descriptionplural={electrically erasable programmable read-only memories},
	firstplural={\glsentrydescplural{EEPROM} (\glsentryplural{EEPROM})}
} 
\newacronym{FB}{FB}{Full Bridge - Vollbrücken}
\newacronym{PCC}{PCC}{Point of common coupling - Netzanschlusspunkt}
\newacronym{BLDC}{BLDC}{Bürstenloser Gleichstrommotor}
\newacronym{PSM}{PSM}{Permanenterregte Synchronmaschiene}
\newacronym{MOSFET}{MOSFET}{Metall Oxid Halbleiter Feldeffekttransistor}
\newacronym{OPV}{OPV}{Operationsverstärker}
\newacronym{VHDL}{VHDL}{Very High Speed Integrated Circuit Hardware Description Language}
\newacronym{IGBT}{IGBT}{Insulated Gate Bipolar Transistor}
%\newacronym{usw.}{usw.}{und so weiter}  % was im Duden steht, muss hier nicht rein
%\newacronym{bzw.}{bzw.}{beziehungsweise}  % was im Duden steht, muss hier nicht rein
%\newacronym{etc.}{etc.}{et cetera}  % was im Duden steht, muss hier nicht rein
%\newacronym{z.B.}{z.B.}{zum Beispiel}  % was im Duden steht, muss hier nicht rein
\newglossaryentry{FAM}{
	name={FAM},
	description={Feldorientierte Regelung der Asynchronmaschine},
	first={\glsentrydesc{FAM} (\glsentrytext{FAM})},
	user1={Feldorientierten Regelung der Asynchronmaschine (\glsentrytext{FAM})}
} 

%--------------------------------------------------------------------
% die Auflistung für das Symbolverzeichnis
%--------------------------------------------------------------------

%% Allgemeine Größen
\newglossaryentry{sym:pollparzahl}{
	name={\ensuremath{p}},
	user6={\glsentrydesc{sym:pollparzahl} \glsentrytext{sym:pollparzahl}},
	description={Polpaarzahl der Asynchronmaschine},
	sort={polPar}, 
	type={symbolslist}
}

\newglossaryentry{sym:inneresMoment}{
	name={\ensuremath{M_\mathrm{i}}},
	user6={\glsentrydesc{sym:inneresMoment} \glsentrytext{sym:inneresMoment}},
	description={inneres Moment der Asynchronmaschine},
	sort={Minner}, 
	type={symbolslist}
}

\newglossaryentry{sym:lastMoment}{
	name={\ensuremath{M_\mathrm{L}}},
	user6={\glsentrydesc{sym:lastMoment} \glsentrytext{sym:lastMoment}},
	description={Lastmoment der Asynchronmaschine},
	sort={Mlast}, 
	type={symbolslist}
}

\newglossaryentry{sym:reibMoment}{
	name={\ensuremath{M_\mathrm{Reib}}},
	user6={\glsentrydesc{sym:reibMoment} \glsentrytext{sym:reibMoment}},
	description={Reibmoment der Asynchronmaschine},
	sort={Mreib}, 
	type={symbolslist}
}

\newglossaryentry{sym:traegheitsmoment}{
	name={\ensuremath{J_\mathrm{Sys}}},
	user6={\glsentrydesc{sym:traegheitsmoment} \glsentrytext{sym:traegheitsmoment}},
	description={Trägheitsmoment des mechanischen Systems},
	sort={Jsys}, 
	type={symbolslist}
}


\newglossaryentry{sym:indukhaupt}{
	name={\ensuremath{L_\mathrm{h}}},
	user6={\glsentrydesc{sym:indukhaupt} \glsentrytext{sym:indukhaupt}},
	description={Hauptinduktivität},
	sort={Lh}, 
	type={symbolslist}
}
\newglossaryentry{sym:magnetisierungsstrom}{
	name={\ensuremath{i_\mathrm{\mu}}},
	user6={\glsentrydesc{sym:magnetisierungsstrom} \glsentrytext{sym:magnetisierungsstrom}},
	description={Magnetisierungsstrom der Asynchronmaschine},
	sort={imagnetstrom}, 
	type={symbolslist}
}

\newglossaryentry{sym:ommegamech}{
	name={\ensuremath{\omega_\mathrm{Mech}}},
	user2= \ensuremath{\omega_\mathrm{Mech}(\Delta t)}, 					% Zeitverlauf der mechanischen Größe 
	user3= \ensuremath{\omega_\mathrm{Mech,0}},
	user4= \ensuremath{\omega_\mathrm{Mech,1}},
	user6={\glsentrydesc{sym:ommegamech} \glsentrytext{sym:ommegamech}},
	description={mechanische Drehfrequenz des Rotors der Asynchronmaschine},
	sort={Omegastator}, 
	type={symbolslist}
}

%% Alle größen für die Statorseite
\newglossaryentry{sym:statorwiderstand}{
	name={\ensuremath{R_\mathrm{S}}},
	user6={\glsentrydesc{sym:statorwiderstand} \glsentrytext{sym:statorwiderstand}},
	description={Statorwiderstand},
	sort={Rs}, 
	type={symbolslist}
}
\newglossaryentry{sym:statorspannungkompl}{
	name={\ensuremath{\uline{u}_\mathrm{S}}},
	user6={\glsentrydesc{sym:statorspannungkompl} \glsentrytext{sym:statorspannungkompl}},
	description={komplexe Statorspannung},
	sort={us}, 
	type={symbolslist}
}
\newglossaryentry{sym:statorstromkompl}{
	name={\ensuremath{\uline{i}_\mathrm{S}}},
	user6={\glsentrydesc{sym:statorstromkompl} \glsentrytext{sym:statorstromkompl}},
	description={komplexer Statorstrom},
	sort={is}, 
	type={symbolslist}
}
\newglossaryentry{sym:statorflusskompl}{
	name={\ensuremath{\uline{\Psi}_\mathrm{S}}},			
	user1={\ensuremath{\uline{\dot{\Psi}}_\mathrm{S}}},			% 1. Ableitung
	user6={\glsentrydesc{sym:statorflusskompl} \glsentrytext{sym:statorflusskompl}},
	description={komplexer Statorfluss},
	sort={Psis}, 
	type={symbolslist}
}
\newglossaryentry{sym:indukstatorstreu}{
	name={\ensuremath{L_\mathrm{S\sigma}}},
	user6={\glsentrydesc{sym:indukstatorstreu} \glsentrytext{sym:indukstatorstreu}},
	description={Statorstreuinduktivität},
	sort={Lssigma}, 
	type={symbolslist}
}

\newglossaryentry{sym:gammarstator}{
	name={\ensuremath{\gamma_\mathrm{S}}},
	user2={\ensuremath{\dot{\gamma}_\mathrm{S}}},	 % Ableitung
	user6={\glsentrydesc{sym:gammarstator} \glsentrytext{sym:gammarstator}},
	description={Winkel des Statorfluss der Asynchronmaschine},
	sort={gammarstator}, 
	type={symbolslist}
}

\newglossaryentry{sym:ommegastator}{
	name={\ensuremath{\omega_\mathrm{S}}},
	user6={\glsentrydesc{sym:ommegastator} \glsentrytext{sym:ommegastator}},
	description={Speißefrequenz des Statorflusses der Asynchronmaschine},
	sort={Omegastator}, 
	type={symbolslist}
}

\newglossaryentry{sym:ommegasyn}{
	name={\ensuremath{\omega_\mathrm{Syn}}},
	user6={\glsentrydesc{sym:ommegastator} \glsentrytext{sym:ommegastator}},
	description={synchrone Kreisfrequenz},
	sort={Omegasyn}, 
	type={symbolslist}
}

\newglossaryentry{sym:ommegaeck}{
	name={\ensuremath{\omega_\mathrm{Eck}}},
	user6={\glsentrydesc{sym:ommegastator} \glsentrytext{sym:ommegastator}},
	description={Eckkreisfrequenz (beginn des Feldschwächebereich)},
	sort={Omegaeck}, 
	type={symbolslist}
}

%% Alle größen für die Rotorseite

\newglossaryentry{sym:rotorwiderstand}{
	name={\ensuremath{R_\mathrm{R}}},
	user2={\ensuremath{R_\mathrm{R}^\prime}},						% Transformiert
	user6={\glsentrydesc{sym:rotorwiderstand} \glsentrytext{sym:rotorwiderstand}},
	description={Rotorwiderstand},
	sort={Rr}, 
	type={symbolslist}
}
\newglossaryentry{sym:rotorspannungkompl}{
	name={\ensuremath{\uline{u}_\mathrm{R}}},
	user6={\glsentrydesc{sym:rotorspannungkompl} \glsentrytext{sym:rotorspannungkompl}},
	description={komplexe Rotorspannung},
	sort={ur}, 
	type={symbolslist}
}
\newglossaryentry{sym:rotorstromkompl}{
	name={\ensuremath{\uline{i}_\mathrm{R}}},
	user2={\ensuremath{\uline{i}_\mathrm{R}^\prime}},				% Transformiert
	user3={\ensuremath{\uline{i}_\mathrm{R}^{\prime*}}},				% konjugiert komplex
	user6={\glsentrydesc{sym:rotorstromkompl} \glsentrytext{sym:rotorstromkompl}},
	description={komplexer Rotorstrom},
	sort={ir}, 
	type={symbolslist}
}
\newglossaryentry{sym:rotorflusskompl}{
	name={\ensuremath{\uline{\Psi}_\mathrm{R}}},			
	user1={\ensuremath{\uline{\dot{\Psi}}_\mathrm{R}}},				% 1. Ableitung
	user2={\ensuremath{\uline{\Psi}_\mathrm{R}^\prime}},			% Transformiert
	user3={\ensuremath{\uline{\dot{\Psi}}_\mathrm{R}^\prime}},		% Transformiert, 1. Ableitung
	user6={\glsentrydesc{sym:rotorflusskompl} \glsentrytext{sym:rotorflusskompl}},
	description={komplexer Rotorfluss},
	sort={Psir}, 
	type={symbolslist}
}
\newglossaryentry{sym:gammarrotor}{
	name={\ensuremath{\gamma_\mathrm{R}}},
	user2={\ensuremath{\dot{\gamma}_\mathrm{R}}},	 % Ableitung
	user6={\glsentrydesc{sym:gammarrotor} \glsentrytext{sym:gammarrotor}},
	description={Winkel des Rotorfluss der Asynchronmaschine},
	sort={gammarrotor}, 
	type={symbolslist}
}

\newglossaryentry{sym:indukrotorstreu}{
	name={\ensuremath{L_\mathrm{R\sigma}}},
	user2={\ensuremath{L_\mathrm{R\sigma}^\prime}},			% Transformiert
	user6={\glsentrydesc{sym:indukrotorstreu} \glsentrytext{sym:indukrotorstreu}},
	description={Rotorstreuinduktivität},
	sort={Lrsigma}, 
	type={symbolslist}
}

\newglossaryentry{sym:ommegarotor}{
	name={\ensuremath{\omega_\mathrm{R}}},
	user6={\glsentrydesc{sym:ommegarotor} \glsentrytext{sym:ommegarotor}},
	description={Speißefrequenz des Rotorflusses der Asynchronmaschine},
	sort={Omegastator}, 
	type={symbolslist}
}
