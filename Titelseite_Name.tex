% Vorlage der Titelseite für Studien- und Diplomarbeiten am ETI
%--------------------------------------------------------------------
%
%   Version: 3.25        (Koma-Skript, documentclass{scrbook})
%   Datum: 01.06.2018
%   Autor: Johannes Bier
%
%--------------------------------------------------------------------

% Logo mit Latex-Standard-Schrift
\thispagestyle{empty}

\begin{textblock}{50}( 17 , 20 )
\includegraphics*[height=1.9cm]{KIT_farb}
\end{textblock}

\begin{textblock}{100}( 95 , 20 )
\begin{flushright}
Elektrotechnisches Institut (ETI)\\
Hybride und Elektrische Fahrzeuge (HEV) / Leistungselektronische Systeme (LES)	% Entsprechende Professur auswählen
\end{flushright}
\end{textblock}


%------ ab hier die eigenen Werte eintragen -----------------------
\begin{textblock}{100}( 95 , 60 )
\begin{flushright}
\large
Karlsruhe, tt.mm.yyyy <- Beginn der Arbeit eintragen    				          	% Beginn der Arbeit eintragen
\end{flushright}
\end{textblock}

\begin{textblock}{170}( 20 , 120 )
\begin{center}
    \begin{LARGE}
    \textsc{Bachelorarbeit}\\										% Art der Arbeit einfügen
    \end{LARGE}
\vspace{0.5cm}
    \begin{huge}
    \textbf{Optimierung elektrischer Maschinen mithilfe iterativer Surrogate Modelle}\\[1cm]           									% Titel der Arbeit
    \end{huge}
\end{center}
\end{textblock}

\begin{textblock}{170}( 20 , 180 )
\begin{center}
\large
Bearbeiter: cand. el. Vorname Name \\           									% eigenen Namen eintragen
\end{center}
\end{textblock}

\begin{textblock}{80}( 20 , 255 )
\large
\noindent
Prof. Dr.-Ing. M. XYZ															% Prof
\end{textblock}

\begin{textblock}{80}( 20 , 260 )
\large
\flushleft
\noindent
Betreuer: Christian Digel                              									% Betreuer
\end{textblock}

\begin{textblock}{80}( 110 , 260 )
\flushright
\large
Abgabetermin: tt.mm.yyyy                        									% Abgabetermin
\end{textblock}

\null\newpage

\thispagestyle{empty}
\newpage

\cleardoublepage
%------------------Erklärung-----------------------------------------
\thispagestyle{empty}
\hspace{2cm}\\
\vspace{16cm}\\
\textbf{Erklärung:}\\
Hiermit versichere ich, dass ich die vorliegende Arbeit selbständig verfasst und keine anderen als die angegebenen Quellen und Hilfsmittel benutzt habe, sowie die wörtlich oder inhaltlich übernommenen Stellen als solche kenntlich gemacht und die Satzung des Karlsruher Institut für Technologie (KIT) zur Sicherung guter wissenschaftlicher Praxis in der jeweils gültigen Fassung beachtet habe.
\vspace*{\fill}

NAME   % Die Erklärung ist zu unterschreiben.

% WICHTIG: Den Wortlaut der Erklärung NICHT ändern!

\null\newpage
\thispagestyle{empty}
\newpage
\cleardoublepage
%------------------Danksagung-----------------------------------------
\thispagestyle{empty}
\hspace{2cm}\\
\vspace{18cm}\\
\textbf{Danksagung}\\
Text der Danksagung\\
%\newpage 
\null\newpage
\thispagestyle{empty}
\newpage
\cleardoublepage
%------------------Kurzfassung-----------------------------------------
% Text von Kurzfassung und Abstract zusammen maximal eine Seite
% Jede Abschlussarbeit sollte einen Abstract/eine Kurzfassung beinhalten.

\thispagestyle{empty}
\section*{Kurzfassung} 
Deutsche Kurzfassung der Arbeit

Der Abstract/Kurzfassung soll den Leser informieren und möglichst neugierig darauf machen, was in der Abschlussarbeit zu erwarten ist. Sinnvoll ist es dabei, explizit die Lesergruppen anzusprechen, für die der Text besonders geeignet ist. Der Abstract ist sprachlich nüchtern und sachlich zu halten. Den Leser interessieren v. a. folgende Fragen: Was sind die wichtigsten Ergebnisse? Welche Methodik wurde wie angewendet? Was sind die wichtigsten Schlussfolgerungen usw.?

Hierbei ist es von Vorteil, wenn Stichwörter wie z.B. 'Multilevel-Umrichter' oder 'Asynchronmaschine' benutzt werden, da es die Suche erleichtert.

\section*{Abstract}
Englische Kurzfassung der Arbeit

\null\newpage
\thispagestyle{empty}
\newpage
\cleardoublepage
