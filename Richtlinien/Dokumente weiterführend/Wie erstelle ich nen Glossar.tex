7.1.17 Wie erstelle ich ein Glossar?

Um ein Wortverzeichnis mit Erkl�rungen oder im weiteren Sinne ein
Verzeichnis von benutzten Begriffen mit Erkl�rungen, wie beispielsweise
von Formelzeichen oder von Abk�rzungen, zu erstellen, bietet LaTeX mit
den Anweisungen \makeglossary und \glossary die M�glichkeit, Eintr�ge
vorzunehmen. Jedoch verbleibt es beim Autor, diese Eintr�ge zu sortieren
und zu formatieren.
Um die Eintr�ge zu sortieren wird meist `makeindex', `xindy' oder
BibTeX, eventuell erg�nzt um ein Zusatzprogramm, verwendet. Zur
Formatierung und zur leichteren oder differenzierteren Eingabe verwendet
man ein entsprechendes LaTeX-Paket.
Das Paket `nomencl' ist entstanden, um Verzeichnisse von Formelzeichen
anzulegen, kann aber dar�berhinaus noch mehr. Datenbanken mit
Abk�rzungen etc. verwaltet das Paket `gloss' mit Hilfe von BibTeX. Das
Tool `glosstex' erstellt automatisch ein Glossar, ein
Abk�rzungsverzeichnis oder ganz allgemein sortierte Listen. Es
kombiniert die Funktionalit�t von `acronym', `nomencl' und `glotex' und
kann ebenfalls Datenbanken verwalten. Mit dem Paket `makeglos' kann in
ein Dokument ein Glossar eingebunden werden, das beispielsweise mit
`makeindex' oder `xindy' erstellt wurde. Dieses Paket ist weniger
umfangreich als `glosstex'.

nomencl:   CTAN: macros/latex/contrib/nomencl
gloss:     CTAN: macros/latex/contrib/gloss
glosstex:  CTAN: support/glosstex/
acronym:   CTAN: macros/latex/contrib/acronym
glotex:    CTAN: indexing/glo+idxtex/
makeglos:  CTAN: macros/latex/contrib/makeglos